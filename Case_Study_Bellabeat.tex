% Options for packages loaded elsewhere
\PassOptionsToPackage{unicode}{hyperref}
\PassOptionsToPackage{hyphens}{url}
%
\documentclass[
]{article}
\usepackage{amsmath,amssymb}
\usepackage{iftex}
\ifPDFTeX
  \usepackage[T1]{fontenc}
  \usepackage[utf8]{inputenc}
  \usepackage{textcomp} % provide euro and other symbols
\else % if luatex or xetex
  \usepackage{unicode-math} % this also loads fontspec
  \defaultfontfeatures{Scale=MatchLowercase}
  \defaultfontfeatures[\rmfamily]{Ligatures=TeX,Scale=1}
\fi
\usepackage{lmodern}
\ifPDFTeX\else
  % xetex/luatex font selection
\fi
% Use upquote if available, for straight quotes in verbatim environments
\IfFileExists{upquote.sty}{\usepackage{upquote}}{}
\IfFileExists{microtype.sty}{% use microtype if available
  \usepackage[]{microtype}
  \UseMicrotypeSet[protrusion]{basicmath} % disable protrusion for tt fonts
}{}
\makeatletter
\@ifundefined{KOMAClassName}{% if non-KOMA class
  \IfFileExists{parskip.sty}{%
    \usepackage{parskip}
  }{% else
    \setlength{\parindent}{0pt}
    \setlength{\parskip}{6pt plus 2pt minus 1pt}}
}{% if KOMA class
  \KOMAoptions{parskip=half}}
\makeatother
\usepackage{xcolor}
\usepackage[margin=1in]{geometry}
\usepackage{color}
\usepackage{fancyvrb}
\newcommand{\VerbBar}{|}
\newcommand{\VERB}{\Verb[commandchars=\\\{\}]}
\DefineVerbatimEnvironment{Highlighting}{Verbatim}{commandchars=\\\{\}}
% Add ',fontsize=\small' for more characters per line
\usepackage{framed}
\definecolor{shadecolor}{RGB}{248,248,248}
\newenvironment{Shaded}{\begin{snugshade}}{\end{snugshade}}
\newcommand{\AlertTok}[1]{\textcolor[rgb]{0.94,0.16,0.16}{#1}}
\newcommand{\AnnotationTok}[1]{\textcolor[rgb]{0.56,0.35,0.01}{\textbf{\textit{#1}}}}
\newcommand{\AttributeTok}[1]{\textcolor[rgb]{0.13,0.29,0.53}{#1}}
\newcommand{\BaseNTok}[1]{\textcolor[rgb]{0.00,0.00,0.81}{#1}}
\newcommand{\BuiltInTok}[1]{#1}
\newcommand{\CharTok}[1]{\textcolor[rgb]{0.31,0.60,0.02}{#1}}
\newcommand{\CommentTok}[1]{\textcolor[rgb]{0.56,0.35,0.01}{\textit{#1}}}
\newcommand{\CommentVarTok}[1]{\textcolor[rgb]{0.56,0.35,0.01}{\textbf{\textit{#1}}}}
\newcommand{\ConstantTok}[1]{\textcolor[rgb]{0.56,0.35,0.01}{#1}}
\newcommand{\ControlFlowTok}[1]{\textcolor[rgb]{0.13,0.29,0.53}{\textbf{#1}}}
\newcommand{\DataTypeTok}[1]{\textcolor[rgb]{0.13,0.29,0.53}{#1}}
\newcommand{\DecValTok}[1]{\textcolor[rgb]{0.00,0.00,0.81}{#1}}
\newcommand{\DocumentationTok}[1]{\textcolor[rgb]{0.56,0.35,0.01}{\textbf{\textit{#1}}}}
\newcommand{\ErrorTok}[1]{\textcolor[rgb]{0.64,0.00,0.00}{\textbf{#1}}}
\newcommand{\ExtensionTok}[1]{#1}
\newcommand{\FloatTok}[1]{\textcolor[rgb]{0.00,0.00,0.81}{#1}}
\newcommand{\FunctionTok}[1]{\textcolor[rgb]{0.13,0.29,0.53}{\textbf{#1}}}
\newcommand{\ImportTok}[1]{#1}
\newcommand{\InformationTok}[1]{\textcolor[rgb]{0.56,0.35,0.01}{\textbf{\textit{#1}}}}
\newcommand{\KeywordTok}[1]{\textcolor[rgb]{0.13,0.29,0.53}{\textbf{#1}}}
\newcommand{\NormalTok}[1]{#1}
\newcommand{\OperatorTok}[1]{\textcolor[rgb]{0.81,0.36,0.00}{\textbf{#1}}}
\newcommand{\OtherTok}[1]{\textcolor[rgb]{0.56,0.35,0.01}{#1}}
\newcommand{\PreprocessorTok}[1]{\textcolor[rgb]{0.56,0.35,0.01}{\textit{#1}}}
\newcommand{\RegionMarkerTok}[1]{#1}
\newcommand{\SpecialCharTok}[1]{\textcolor[rgb]{0.81,0.36,0.00}{\textbf{#1}}}
\newcommand{\SpecialStringTok}[1]{\textcolor[rgb]{0.31,0.60,0.02}{#1}}
\newcommand{\StringTok}[1]{\textcolor[rgb]{0.31,0.60,0.02}{#1}}
\newcommand{\VariableTok}[1]{\textcolor[rgb]{0.00,0.00,0.00}{#1}}
\newcommand{\VerbatimStringTok}[1]{\textcolor[rgb]{0.31,0.60,0.02}{#1}}
\newcommand{\WarningTok}[1]{\textcolor[rgb]{0.56,0.35,0.01}{\textbf{\textit{#1}}}}
\usepackage{graphicx}
\makeatletter
\def\maxwidth{\ifdim\Gin@nat@width>\linewidth\linewidth\else\Gin@nat@width\fi}
\def\maxheight{\ifdim\Gin@nat@height>\textheight\textheight\else\Gin@nat@height\fi}
\makeatother
% Scale images if necessary, so that they will not overflow the page
% margins by default, and it is still possible to overwrite the defaults
% using explicit options in \includegraphics[width, height, ...]{}
\setkeys{Gin}{width=\maxwidth,height=\maxheight,keepaspectratio}
% Set default figure placement to htbp
\makeatletter
\def\fps@figure{htbp}
\makeatother
\setlength{\emergencystretch}{3em} % prevent overfull lines
\providecommand{\tightlist}{%
  \setlength{\itemsep}{0pt}\setlength{\parskip}{0pt}}
\setcounter{secnumdepth}{-\maxdimen} % remove section numbering
\usepackage{booktabs}
\usepackage{longtable}
\usepackage{array}
\usepackage{multirow}
\usepackage{wrapfig}
\usepackage{float}
\usepackage{colortbl}
\usepackage{pdflscape}
\usepackage{tabu}
\usepackage{threeparttable}
\usepackage{threeparttablex}
\usepackage[normalem]{ulem}
\usepackage{makecell}
\usepackage{xcolor}
\ifLuaTeX
  \usepackage{selnolig}  % disable illegal ligatures
\fi
\usepackage{bookmark}
\IfFileExists{xurl.sty}{\usepackage{xurl}}{} % add URL line breaks if available
\urlstyle{same}
\hypersetup{
  pdftitle={Case Study Bellabeat},
  pdfauthor={Akilan Sivanandham},
  hidelinks,
  pdfcreator={LaTeX via pandoc}}

\title{Case Study Bellabeat}
\author{Akilan Sivanandham}
\date{2024-06-02}

\begin{document}
\maketitle

\section{Introduction}\label{introduction}

Bellabeat, a leading high-tech manufacturer of health-focused products
for women, aims to answer key business questions by understanding and
analyzing relevant data. We will use data from the FitBit Fitness
Tracker, collected through a dispersed survey conducted from December 3,
2016, to December 5, 2016, via Amazon Mechanical Turk. Thirty Fitbit
users who met the eligibility requirements consented to submit their
personal tracker data, which includes minute-by-minute output for heart
rate, physical activity, and sleep tracking.

We will clean and manipulate this dataset to ensure accurate and
reliable analysis. Our approach will involve creating visualizations and
employing statistical models to identify trends and patterns within the
data.

The analysis will provide actionable insights for Bellabeat, offering
recommendations on how to refine its marketing strategy and enhance
product development. Ultimately, our goal is to help Bellabeat better
understand its customers and the market, enabling data-driven decisions
on the following questions.

\subsubsection{Our main focus on following
questions}\label{our-main-focus-on-following-questions}

\begin{itemize}
\tightlist
\item
  What are the primary patterns in Bellabeat smart device utlization?
\item
  How can the above insights enhance Bellabeat on its marketing strategy
  and product development?
\item
  What are the main opportunities and challenges that Bellabeat faces in
  the smart device industry?
\end{itemize}

\subsubsection{ASK}\label{ask}

\subparagraph{Task}\label{task}

Sršen has tasked you with analyzing smart device usage data to gain
insights into how consumers use non-Bellabeat smart devices. Based on
these insights, you are to select one Bellabeat product to focus on in
your presentation

\#\#\#Prepare

\paragraph{Loading packages}\label{loading-packages}

In order to proceed further we need to looad following packages in our
R-Studio environment

\begin{Shaded}
\begin{Highlighting}[]
\FunctionTok{library}\NormalTok{(tidyverse)}
\end{Highlighting}
\end{Shaded}

\begin{verbatim}
## -- Attaching core tidyverse packages ------------------------ tidyverse 2.0.0 --
## v dplyr     1.1.4     v readr     2.1.5
## v forcats   1.0.0     v stringr   1.5.1
## v ggplot2   3.5.1     v tibble    3.2.1
## v lubridate 1.9.3     v tidyr     1.3.1
## v purrr     1.0.2     
## -- Conflicts ------------------------------------------ tidyverse_conflicts() --
## x dplyr::filter() masks stats::filter()
## x dplyr::lag()    masks stats::lag()
## i Use the conflicted package (<http://conflicted.r-lib.org/>) to force all conflicts to become errors
\end{verbatim}

\begin{Shaded}
\begin{Highlighting}[]
\FunctionTok{library}\NormalTok{(lubridate)}
\FunctionTok{library}\NormalTok{(ggplot2)}
\FunctionTok{library}\NormalTok{(dplyr)}
\FunctionTok{library}\NormalTok{(stringr)}
\FunctionTok{library}\NormalTok{(highcharter)}
\end{Highlighting}
\end{Shaded}

\begin{verbatim}
## Registered S3 method overwritten by 'quantmod':
##   method            from
##   as.zoo.data.frame zoo 
## Highcharts (www.highcharts.com) is a Highsoft software product which is
## not free for commercial and Governmental use
\end{verbatim}

\begin{Shaded}
\begin{Highlighting}[]
\FunctionTok{library}\NormalTok{(janitor)}
\end{Highlighting}
\end{Shaded}

\begin{verbatim}
## 
## Attaching package: 'janitor'
## 
## The following objects are masked from 'package:stats':
## 
##     chisq.test, fisher.test
\end{verbatim}

\begin{Shaded}
\begin{Highlighting}[]
\FunctionTok{library}\NormalTok{(RColorBrewer)}
\FunctionTok{library}\NormalTok{(kableExtra)}
\end{Highlighting}
\end{Shaded}

\begin{verbatim}
## 
## Attaching package: 'kableExtra'
## 
## The following object is masked from 'package:dplyr':
## 
##     group_rows
\end{verbatim}

\begin{Shaded}
\begin{Highlighting}[]
\FunctionTok{library}\NormalTok{(scales)}
\end{Highlighting}
\end{Shaded}

\begin{verbatim}
## 
## Attaching package: 'scales'
## 
## The following object is masked from 'package:purrr':
## 
##     discard
## 
## The following object is masked from 'package:readr':
## 
##     col_factor
\end{verbatim}

\begin{Shaded}
\begin{Highlighting}[]
\FunctionTok{library}\NormalTok{(readr)}
\end{Highlighting}
\end{Shaded}

\paragraph{Loading data}\label{loading-data}

uploadig data in to dataset. due to number of files, we use following
method to load all the data and create dataset with appropriate names.

\begin{Shaded}
\begin{Highlighting}[]
\CommentTok{\# Set the working directory which contain the download CSV files}
\FunctionTok{setwd}\NormalTok{(}\StringTok{\textquotesingle{}Data/mturkfitbit\_export\_4.12.16{-}5.12.16/Fitabase Data 4.12.16{-}5.12.16\textquotesingle{}}\NormalTok{)}

\CommentTok{\# Get all the CSV files names from the folder.}
\NormalTok{csv\_files }\OtherTok{\textless{}{-}} \FunctionTok{list.files}\NormalTok{(}\AttributeTok{pattern =} \StringTok{\textquotesingle{}*.csv\textquotesingle{}}\NormalTok{)}

\CommentTok{\#Read CSV files functions}
\NormalTok{read\_csv\_files }\OtherTok{\textless{}{-}} \ControlFlowTok{function}\NormalTok{(file\_name)\{}
  \FunctionTok{read\_csv}\NormalTok{(file\_name)}
\NormalTok{\}}

\NormalTok{data\_frame }\OtherTok{\textless{}{-}} \FunctionTok{lapply}\NormalTok{(csv\_files, read\_csv\_files)}
\end{Highlighting}
\end{Shaded}

\begin{verbatim}
## Rows: 940 Columns: 15
## -- Column specification --------------------------------------------------------
## Delimiter: ","
## chr  (1): ActivityDate
## dbl (14): Id, TotalSteps, TotalDistance, TrackerDistance, LoggedActivitiesDi...
## 
## i Use `spec()` to retrieve the full column specification for this data.
## i Specify the column types or set `show_col_types = FALSE` to quiet this message.
## Rows: 940 Columns: 3
## -- Column specification --------------------------------------------------------
## Delimiter: ","
## chr (1): ActivityDay
## dbl (2): Id, Calories
## 
## i Use `spec()` to retrieve the full column specification for this data.
## i Specify the column types or set `show_col_types = FALSE` to quiet this message.
## Rows: 940 Columns: 10
## -- Column specification --------------------------------------------------------
## Delimiter: ","
## chr (1): ActivityDay
## dbl (9): Id, SedentaryMinutes, LightlyActiveMinutes, FairlyActiveMinutes, Ve...
## 
## i Use `spec()` to retrieve the full column specification for this data.
## i Specify the column types or set `show_col_types = FALSE` to quiet this message.
## Rows: 940 Columns: 3
## -- Column specification --------------------------------------------------------
## Delimiter: ","
## chr (1): ActivityDay
## dbl (2): Id, StepTotal
## 
## i Use `spec()` to retrieve the full column specification for this data.
## i Specify the column types or set `show_col_types = FALSE` to quiet this message.
## Rows: 2483658 Columns: 3
## -- Column specification --------------------------------------------------------
## Delimiter: ","
## chr (1): Time
## dbl (2): Id, Value
## 
## i Use `spec()` to retrieve the full column specification for this data.
## i Specify the column types or set `show_col_types = FALSE` to quiet this message.
## Rows: 22099 Columns: 3
## -- Column specification --------------------------------------------------------
## Delimiter: ","
## chr (1): ActivityHour
## dbl (2): Id, Calories
## 
## i Use `spec()` to retrieve the full column specification for this data.
## i Specify the column types or set `show_col_types = FALSE` to quiet this message.
## Rows: 22099 Columns: 4
## -- Column specification --------------------------------------------------------
## Delimiter: ","
## chr (1): ActivityHour
## dbl (3): Id, TotalIntensity, AverageIntensity
## 
## i Use `spec()` to retrieve the full column specification for this data.
## i Specify the column types or set `show_col_types = FALSE` to quiet this message.
## Rows: 22099 Columns: 3
## -- Column specification --------------------------------------------------------
## Delimiter: ","
## chr (1): ActivityHour
## dbl (2): Id, StepTotal
## 
## i Use `spec()` to retrieve the full column specification for this data.
## i Specify the column types or set `show_col_types = FALSE` to quiet this message.
## Rows: 413 Columns: 5
## -- Column specification --------------------------------------------------------
## Delimiter: ","
## chr (1): SleepDay
## dbl (4): Id, TotalSleepRecords, TotalMinutesAsleep, TotalTimeInBed
## 
## i Use `spec()` to retrieve the full column specification for this data.
## i Specify the column types or set `show_col_types = FALSE` to quiet this message.
## Rows: 67 Columns: 8
## -- Column specification --------------------------------------------------------
## Delimiter: ","
## chr (1): Date
## dbl (6): Id, WeightKg, WeightPounds, Fat, BMI, LogId
## lgl (1): IsManualReport
## 
## i Use `spec()` to retrieve the full column specification for this data.
## i Specify the column types or set `show_col_types = FALSE` to quiet this message.
\end{verbatim}

\begin{Shaded}
\begin{Highlighting}[]
\CommentTok{\#Create Data set name from the file}

\NormalTok{create\_dataset\_name }\OtherTok{\textless{}{-}} \ControlFlowTok{function}\NormalTok{(file\_name)}
\NormalTok{\{}
\NormalTok{   name }\OtherTok{\textless{}{-}}\NormalTok{ tools}\SpecialCharTok{::}\FunctionTok{file\_path\_sans\_ext}\NormalTok{(}\FunctionTok{basename}\NormalTok{(file\_name))}

   \CommentTok{\# Split the name by underscores}
\NormalTok{  name\_parts }\OtherTok{\textless{}{-}} \FunctionTok{unlist}\NormalTok{(}\FunctionTok{strsplit}\NormalTok{(name, }\StringTok{"\_"}\NormalTok{))}
  
  \CommentTok{\# Remove the last part if there are more than one part}
  \ControlFlowTok{if}\NormalTok{ (}\FunctionTok{length}\NormalTok{(name\_parts) }\SpecialCharTok{\textgreater{}} \DecValTok{1}\NormalTok{) \{}
\NormalTok{    name\_parts }\OtherTok{\textless{}{-}}\NormalTok{ name\_parts[}\SpecialCharTok{{-}}\FunctionTok{length}\NormalTok{(name\_parts)]}
\NormalTok{  \}}
  
  \CommentTok{\# Combine the remaining parts}
\NormalTok{  name }\OtherTok{\textless{}{-}} \FunctionTok{paste}\NormalTok{(name\_parts, }\AttributeTok{collapse =} \StringTok{"\_"}\NormalTok{)}
   
   
 \CommentTok{\# Insert underscores before capital letters and lowercase letters followed by capital letters}
\NormalTok{  name }\OtherTok{\textless{}{-}} \FunctionTok{gsub}\NormalTok{(}\StringTok{"([a{-}z])([A{-}Z])"}\NormalTok{, }\StringTok{"}\SpecialCharTok{\textbackslash{}\textbackslash{}}\StringTok{1\_}\SpecialCharTok{\textbackslash{}\textbackslash{}}\StringTok{2"}\NormalTok{, name)    }\CommentTok{\# Insert underscore between lowercase and uppercase}
\NormalTok{  name }\OtherTok{\textless{}{-}} \FunctionTok{gsub}\NormalTok{(}\StringTok{"([A{-}Z])([A{-}Z][a{-}z])"}\NormalTok{, }\StringTok{"}\SpecialCharTok{\textbackslash{}\textbackslash{}}\StringTok{1\_}\SpecialCharTok{\textbackslash{}\textbackslash{}}\StringTok{2"}\NormalTok{, name) }\CommentTok{\# Insert underscore before subsequent capital letters}
  
  \CommentTok{\# Ensure all parts are capitalized correctly and join with underscores}
\NormalTok{  name\_parts }\OtherTok{\textless{}{-}} \FunctionTok{unlist}\NormalTok{(}\FunctionTok{strsplit}\NormalTok{(name, }\StringTok{"\_"}\NormalTok{))}
\NormalTok{  name }\OtherTok{\textless{}{-}} \FunctionTok{paste}\NormalTok{(name\_parts, }\AttributeTok{collapse =} \StringTok{"\_"}\NormalTok{)}
  
\NormalTok{  name}
\NormalTok{\}}

\CommentTok{\# Assign data frames to dynamically created names}
\ControlFlowTok{for}\NormalTok{ (i }\ControlFlowTok{in} \FunctionTok{seq\_along}\NormalTok{(csv\_files)) \{}
\NormalTok{  dataset\_name }\OtherTok{\textless{}{-}} \FunctionTok{create\_dataset\_name}\NormalTok{(csv\_files[i])}
  \FunctionTok{assign}\NormalTok{(dataset\_name, data\_frame[[i]], }\AttributeTok{envir =}\NormalTok{ .GlobalEnv)}
\NormalTok{\}}


\CommentTok{\# List the names of the created datasets}
\NormalTok{created\_datasets }\OtherTok{\textless{}{-}} \FunctionTok{sapply}\NormalTok{(csv\_files, create\_dataset\_name)}
\FunctionTok{print}\NormalTok{(created\_datasets)}
\end{Highlighting}
\end{Shaded}

\begin{verbatim}
##     dailyActivity_merged.csv     dailyCalories_merged.csv 
##             "daily_Activity"             "daily_Calories" 
##  dailyIntensities_merged.csv        dailySteps_merged.csv 
##          "daily_Intensities"                "daily_Steps" 
## heartrate_seconds_merged.csv    hourlyCalories_merged.csv 
##          "heartrate_seconds"            "hourly_Calories" 
## hourlyIntensities_merged.csv       hourlySteps_merged.csv 
##         "hourly_Intensities"               "hourly_Steps" 
##          sleepDay_merged.csv     weightLogInfo_merged.csv 
##                  "sleep_Day"            "weight_Log_Info"
\end{verbatim}

\begin{Shaded}
\begin{Highlighting}[]
\FunctionTok{head}\NormalTok{(daily\_Activity)}
\end{Highlighting}
\end{Shaded}

\begin{verbatim}
## # A tibble: 6 x 15
##           Id ActivityDate TotalSteps TotalDistance TrackerDistance
##        <dbl> <chr>             <dbl>         <dbl>           <dbl>
## 1 1503960366 4/12/2016         13162          8.5             8.5 
## 2 1503960366 4/13/2016         10735          6.97            6.97
## 3 1503960366 4/14/2016         10460          6.74            6.74
## 4 1503960366 4/15/2016          9762          6.28            6.28
## 5 1503960366 4/16/2016         12669          8.16            8.16
## 6 1503960366 4/17/2016          9705          6.48            6.48
## # i 10 more variables: LoggedActivitiesDistance <dbl>,
## #   VeryActiveDistance <dbl>, ModeratelyActiveDistance <dbl>,
## #   LightActiveDistance <dbl>, SedentaryActiveDistance <dbl>,
## #   VeryActiveMinutes <dbl>, FairlyActiveMinutes <dbl>,
## #   LightlyActiveMinutes <dbl>, SedentaryMinutes <dbl>, Calories <dbl>
\end{verbatim}

\subsubsection{Process}\label{process}

\begin{Shaded}
\begin{Highlighting}[]
\CommentTok{\# Glimpse of all datasets}
\ControlFlowTok{for}\NormalTok{ (dataset\_name }\ControlFlowTok{in}\NormalTok{ created\_datasets) \{}
  \FunctionTok{cat}\NormalTok{(}\StringTok{"Dataset:"}\NormalTok{, dataset\_name, }\StringTok{"}\SpecialCharTok{\textbackslash{}n}\StringTok{"}\NormalTok{)}
  \FunctionTok{glimpse}\NormalTok{(}\FunctionTok{get}\NormalTok{(dataset\_name))}
  \FunctionTok{cat}\NormalTok{(}\StringTok{"}\SpecialCharTok{\textbackslash{}n\textbackslash{}n}\StringTok{"}\NormalTok{)}
\NormalTok{\}}
\end{Highlighting}
\end{Shaded}

\begin{verbatim}
## Dataset: daily_Activity 
## Rows: 940
## Columns: 15
## $ Id                       <dbl> 1503960366, 1503960366, 1503960366, 150396036~
## $ ActivityDate             <chr> "4/12/2016", "4/13/2016", "4/14/2016", "4/15/~
## $ TotalSteps               <dbl> 13162, 10735, 10460, 9762, 12669, 9705, 13019~
## $ TotalDistance            <dbl> 8.50, 6.97, 6.74, 6.28, 8.16, 6.48, 8.59, 9.8~
## $ TrackerDistance          <dbl> 8.50, 6.97, 6.74, 6.28, 8.16, 6.48, 8.59, 9.8~
## $ LoggedActivitiesDistance <dbl> 0, 0, 0, 0, 0, 0, 0, 0, 0, 0, 0, 0, 0, 0, 0, ~
## $ VeryActiveDistance       <dbl> 1.88, 1.57, 2.44, 2.14, 2.71, 3.19, 3.25, 3.5~
## $ ModeratelyActiveDistance <dbl> 0.55, 0.69, 0.40, 1.26, 0.41, 0.78, 0.64, 1.3~
## $ LightActiveDistance      <dbl> 6.06, 4.71, 3.91, 2.83, 5.04, 2.51, 4.71, 5.0~
## $ SedentaryActiveDistance  <dbl> 0, 0, 0, 0, 0, 0, 0, 0, 0, 0, 0, 0, 0, 0, 0, ~
## $ VeryActiveMinutes        <dbl> 25, 21, 30, 29, 36, 38, 42, 50, 28, 19, 66, 4~
## $ FairlyActiveMinutes      <dbl> 13, 19, 11, 34, 10, 20, 16, 31, 12, 8, 27, 21~
## $ LightlyActiveMinutes     <dbl> 328, 217, 181, 209, 221, 164, 233, 264, 205, ~
## $ SedentaryMinutes         <dbl> 728, 776, 1218, 726, 773, 539, 1149, 775, 818~
## $ Calories                 <dbl> 1985, 1797, 1776, 1745, 1863, 1728, 1921, 203~
## 
## 
## Dataset: daily_Calories 
## Rows: 940
## Columns: 3
## $ Id          <dbl> 1503960366, 1503960366, 1503960366, 1503960366, 1503960366~
## $ ActivityDay <chr> "4/12/2016", "4/13/2016", "4/14/2016", "4/15/2016", "4/16/~
## $ Calories    <dbl> 1985, 1797, 1776, 1745, 1863, 1728, 1921, 2035, 1786, 1775~
## 
## 
## Dataset: daily_Intensities 
## Rows: 940
## Columns: 10
## $ Id                       <dbl> 1503960366, 1503960366, 1503960366, 150396036~
## $ ActivityDay              <chr> "4/12/2016", "4/13/2016", "4/14/2016", "4/15/~
## $ SedentaryMinutes         <dbl> 728, 776, 1218, 726, 773, 539, 1149, 775, 818~
## $ LightlyActiveMinutes     <dbl> 328, 217, 181, 209, 221, 164, 233, 264, 205, ~
## $ FairlyActiveMinutes      <dbl> 13, 19, 11, 34, 10, 20, 16, 31, 12, 8, 27, 21~
## $ VeryActiveMinutes        <dbl> 25, 21, 30, 29, 36, 38, 42, 50, 28, 19, 66, 4~
## $ SedentaryActiveDistance  <dbl> 0, 0, 0, 0, 0, 0, 0, 0, 0, 0, 0, 0, 0, 0, 0, ~
## $ LightActiveDistance      <dbl> 6.06, 4.71, 3.91, 2.83, 5.04, 2.51, 4.71, 5.0~
## $ ModeratelyActiveDistance <dbl> 0.55, 0.69, 0.40, 1.26, 0.41, 0.78, 0.64, 1.3~
## $ VeryActiveDistance       <dbl> 1.88, 1.57, 2.44, 2.14, 2.71, 3.19, 3.25, 3.5~
## 
## 
## Dataset: daily_Steps 
## Rows: 940
## Columns: 3
## $ Id          <dbl> 1503960366, 1503960366, 1503960366, 1503960366, 1503960366~
## $ ActivityDay <chr> "4/12/2016", "4/13/2016", "4/14/2016", "4/15/2016", "4/16/~
## $ StepTotal   <dbl> 13162, 10735, 10460, 9762, 12669, 9705, 13019, 15506, 1054~
## 
## 
## Dataset: heartrate_seconds 
## Rows: 2,483,658
## Columns: 3
## $ Id    <dbl> 2022484408, 2022484408, 2022484408, 2022484408, 2022484408, 2022~
## $ Time  <chr> "4/12/2016 7:21:00 AM", "4/12/2016 7:21:05 AM", "4/12/2016 7:21:~
## $ Value <dbl> 97, 102, 105, 103, 101, 95, 91, 93, 94, 93, 92, 89, 83, 61, 60, ~
## 
## 
## Dataset: hourly_Calories 
## Rows: 22,099
## Columns: 3
## $ Id           <dbl> 1503960366, 1503960366, 1503960366, 1503960366, 150396036~
## $ ActivityHour <chr> "4/12/2016 12:00:00 AM", "4/12/2016 1:00:00 AM", "4/12/20~
## $ Calories     <dbl> 81, 61, 59, 47, 48, 48, 48, 47, 68, 141, 99, 76, 73, 66, ~
## 
## 
## Dataset: hourly_Intensities 
## Rows: 22,099
## Columns: 4
## $ Id               <dbl> 1503960366, 1503960366, 1503960366, 1503960366, 15039~
## $ ActivityHour     <chr> "4/12/2016 12:00:00 AM", "4/12/2016 1:00:00 AM", "4/1~
## $ TotalIntensity   <dbl> 20, 8, 7, 0, 0, 0, 0, 0, 13, 30, 29, 12, 11, 6, 36, 5~
## $ AverageIntensity <dbl> 0.333333, 0.133333, 0.116667, 0.000000, 0.000000, 0.0~
## 
## 
## Dataset: hourly_Steps 
## Rows: 22,099
## Columns: 3
## $ Id           <dbl> 1503960366, 1503960366, 1503960366, 1503960366, 150396036~
## $ ActivityHour <chr> "4/12/2016 12:00:00 AM", "4/12/2016 1:00:00 AM", "4/12/20~
## $ StepTotal    <dbl> 373, 160, 151, 0, 0, 0, 0, 0, 250, 1864, 676, 360, 253, 2~
## 
## 
## Dataset: sleep_Day 
## Rows: 413
## Columns: 5
## $ Id                 <dbl> 1503960366, 1503960366, 1503960366, 1503960366, 150~
## $ SleepDay           <chr> "4/12/2016 12:00:00 AM", "4/13/2016 12:00:00 AM", "~
## $ TotalSleepRecords  <dbl> 1, 2, 1, 2, 1, 1, 1, 1, 1, 1, 1, 1, 1, 1, 1, 1, 1, ~
## $ TotalMinutesAsleep <dbl> 327, 384, 412, 340, 700, 304, 360, 325, 361, 430, 2~
## $ TotalTimeInBed     <dbl> 346, 407, 442, 367, 712, 320, 377, 364, 384, 449, 3~
## 
## 
## Dataset: weight_Log_Info 
## Rows: 67
## Columns: 8
## $ Id             <dbl> 1503960366, 1503960366, 1927972279, 2873212765, 2873212~
## $ Date           <chr> "5/2/2016 11:59:59 PM", "5/3/2016 11:59:59 PM", "4/13/2~
## $ WeightKg       <dbl> 52.6, 52.6, 133.5, 56.7, 57.3, 72.4, 72.3, 69.7, 70.3, ~
## $ WeightPounds   <dbl> 115.9631, 115.9631, 294.3171, 125.0021, 126.3249, 159.6~
## $ Fat            <dbl> 22, NA, NA, NA, NA, 25, NA, NA, NA, NA, NA, NA, NA, NA,~
## $ BMI            <dbl> 22.65, 22.65, 47.54, 21.45, 21.69, 27.45, 27.38, 27.25,~
## $ IsManualReport <lgl> TRUE, TRUE, FALSE, TRUE, TRUE, TRUE, TRUE, TRUE, TRUE, ~
## $ LogId          <dbl> 1.462234e+12, 1.462320e+12, 1.460510e+12, 1.461283e+12,~
\end{verbatim}

\paragraph{Split the Date and time for
Datasets.}\label{split-the-date-and-time-for-datasets.}

In order to process further, we need to split date and time from the
following dataset in differnt column. * heartrate\_seconds - Dataset
about the information of heart rate by seconds * hourly\_Calories -
Dataset about the information of Calories by hourly *
hourly\_Intensities - Dataset about the information of Intensities by
hourly * hourly\_Steps - Dataset about the information of Steps by
hourly * sleep\_Day - Dataset about the information of Sleeping hours by
day * weight\_Log\_Info - Dataset about the information of Weight

\begin{Shaded}
\begin{Highlighting}[]
\FunctionTok{head}\NormalTok{(heartrate\_seconds)}
\end{Highlighting}
\end{Shaded}

\begin{verbatim}
## # A tibble: 6 x 3
##           Id Time                 Value
##        <dbl> <chr>                <dbl>
## 1 2022484408 4/12/2016 7:21:00 AM    97
## 2 2022484408 4/12/2016 7:21:05 AM   102
## 3 2022484408 4/12/2016 7:21:10 AM   105
## 4 2022484408 4/12/2016 7:21:20 AM   103
## 5 2022484408 4/12/2016 7:21:25 AM   101
## 6 2022484408 4/12/2016 7:22:05 AM    95
\end{verbatim}

\end{document}
